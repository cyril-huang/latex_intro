\chapter{進階使用}
上面的基本介紹應該足夠知識了解粗淺的{\LaTeX} 系統,
但還是有很多細微龜毛調整是我們要知道的,尤其前面我們完全不管
一些粗細長寬等設定,只專注在文章邏輯結構,章節目錄等等。
可是它的長相都是 {\LaTeX} 內定的,當我們想讓本文寬度更大些就要
來改變內定設定。 最後要來知道 macro 是怎麼寫出來,將來我們要自己增加功能。
\\\\
要先了知的是整個版面的相關術語與知識,然後其實電子排版就是在一張虛擬畫布
上,把物件擺來擺去,所以設定各物件空白的命令 macro 很重要。最後我們要能
用原始命令增加自己的新命令。

\section{參數設定}
包括一些內定counter,點線面的設定。
\subsection{幾何設定}
排版對於字型的定義是很嚴格的,以下為一個字的定義
\begin{center}
\includegraphics[width=0.5\textwidth,keepaspectratio]{images/sphinx.png}
\end{center}
大寫字一律要站在 baseline,很多的對齊參考基準都是根據這些定義而來丈量的。
\\\\
設定直線或黑盒子,直線就是一塊細長的黑盒子
\begin{center}
\begin{tabular}{ccc}
命令 & 效果 & 意義 \\
\hline
\verb=\rule[-1mm]{5mm}{3mm}= & \rule[-1mm]{5mm}{3mm} Sphinx & 
   從baseline下1mm 升起5mm寬3mm厚的黑盒子\\
\verb=\rule{5mm}{.1pt}= & \rule{5mm}{.1pt} Sphinx &
   從baseline升起5mm寬.1pt厚的黑盒子\\
\end{tabular}
\end{center}
設定文字盒子
\begin{center}
\begin{tabular}{cm{4cm}m{4cm}}
命令 & 效果 & 意義 \\
\hline
\verb=\parbox{4cm}{paragraph 盒子}= &
  \parbox{4cm}{paragraph 盒子如果文字長度太長,會自動折行,
  跟 tabular 的 p 設定一樣。} & 段落盒子 \\
\verb=\mbox{mbox}= & \mbox{mbox} & 無框盒子 \\
\verb=\makebox[3cm][s]{make a box}= & \makebox[3cm][s]{make a box} & 寬3cm,文字spread盒子 \\
\verb=\fbox{fbox}= & \fbox{fbox} & 有框盒子 \\
\verb=\framebox[3cm][r]{frame a box}= & \framebox[3cm][r]{frame a box} & 寬3cm,文字從右邊寫起 \\
\end{tabular}
\end{center}
設定某參數長度在 preamble 區使用 setlength,它會影響全部,例如
\begin{verbatim}
\documentclass{book}
\setlength{\hoffset}{0mm}
\setlength{\oddsidemargin}{20mm-1in}
\setlength{\paperwidth}{210mm}
\setlength{\textwidth}{\paperwidth-40mm}
\begin{document}
...
\end{verbatim}
{\LaTeX}設新長度名字方法,其中可以用一群文字來把
新變數的寬度高度深度設成後面文字的寬度高度深度。
\begin{verbatim}
\newlength{\mylength}
\setlength{\mylength}{100cm}
\settowidth{\mylength}{大頭 大頭你好長}
\settoheight{\mylength}{大雄 大雄你好高}
\settodepth{\mylength}{大家 大家你好深}
\end{verbatim}
原本 {\TeX} 的方法
\begin{verbatim}
\newdimen\mylength
mylength=1.5pt
\end{verbatim}
印出長度用the 會印出。
\begin{center}
\begin{tabular}{ccc}
命令 & 效果 & 意義\\
\hline
\verb=\the\textwidth= & \the\textwidth & 文字區域寬度 \\
\verb=\the\paperwidth= & \the\paperwidth & 紙張真正寬度 \\
\verb=\the\paperheight= & \the\paperheight & 紙張真正高度 \\
\verb=\the\tabcolsep= & \the\tabcolsep & 表格內欄位間距離 \\
\end{tabular}
\end{center}
這裡面的單位除了簡單的公英制單位外
\begin{itemize}
\item pt 前面說過大約是0.35mm
\item em 這是目前使用字型M的寬度
\item ex 這是目前使用字型x的高度
\end{itemize}

\subsection{版面設定}
layout 是整個版面的定義,長寬高各種術語與設定。
先用 layout package 的 \verb=\layout{}= 印出來如下
\clearpage
\layout
一頁是奇數,一頁是偶數,這個要用 geometry package 來設定。
\begin{verbatim}
\usepackage[a4paper]{geometry}

\usepackage[
  paperwidth=5.5in, 
  paperheight=8.5in]
{geometry}
\end{verbatim}
還有一些簡單的設定
\begin{verbatim}
\usepackage[
  top=1in,
  bottom=1.25in,
  left=1.25in,
  right=1.25in]
{geometry}

\usepackage[margin=1in]{geometry}
\end{verbatim}
表示設定四周紙張到文字的距離,用margin=2cm 是比較快速設定四周距離。

\subsection{\LaTeX 空白設定}
基本小空白設定
\begin{itemize}
\item \verb=~= 多出一個小空白
\item \verb=\hspace{1cm}= 向右多出1cm空白,負數向左。
\item \verb=\vspace{1cm}= 向下空出1cm空白,負數向上。
\item \verb=\quad \qquad= 向右空出1em 與 2em 的空白。
\end{itemize}
行間的空白, 要注意的是 big, med, small 這些 skip 在隔開文字時,只有在空白
一行段落後面才是對的。 因為它們是垂直的,所以只能是垂直物件間的距離。
\begin{itemize}
\item \verb=\bigskip= 產生 12pt 的垂直空白。
\item \verb=\medskip= 產生 6pt 的垂直空白。
\item \verb=\smallskip= 產生 3pt 的垂直空白。
\end{itemize}
多空白的產生
\begin{itemize}
\item \verb=\stretch{1.5}= 產生 1.5 倍原有的空白,例如\verb=\hspace{\stretch{3}}=
\item \verb=\linespread{1.5}= 產生原本 1.5 倍的行距空白。
\item \verb=\hfill= 如果這段文字寬度不夠充滿整行,就從 hfill 開始填空白使得
這段文字充滿一行,這跟前面說的用 makebox 的 spread 不一樣,只是單純分開文字。
\item \verb=\vfill= 類似 hfill,但 vfill 兩邊的字填上空白直到佔滿整頁。
\end{itemize} 
paragraph 段落對齊設定,這就是在 一般 word 等軟體看到的三種對齊,
如果文字不夠一行, 這會自動補上本文寬度的前後空白。例如
\begin{flushright}
哈哈 MS Word: 我可以對齊右邊耶
\end{flushright}
\begin{center}
\begin{tabular}{ccc}
對齊 & 環境命令  & 基本命令\\
\hline
左對齊 & flushleft & \verb=\raggedright= \\
右對齊 & flushright & \verb=\raggedleft= \\
中對齊 & center & \verb=\centering= \\
\end{tabular}
\end{center}
段落空白
\begin{itemize}
\item \verb=\parindent= 段落縮排時的空白。
\item \verb=\parskip= 每個段落間距離。
\item \verb=\indent= 恢復縮排 indent
\item \verb=\noindent= 不要縮排 indent
\end{itemize}
整頁空白,有時在大文件時或者有些 floating 物件讓頁面看起來排的不好
\begin{itemize}
\item \verb=\newpage= 結束這一頁,另起新頁。
\item \verb=\pagebreak[num]= 在這命令之後  break 這一頁。num 是 0-4 的 priorty 。
\item \verb=\clearpage= 結束這一頁,而且任何在這頁前的 float 物件印出。
之後的新 float 物件都重新計算,從新頁開始,
這個對付一些跑來跑去的大物件很有效。前面兩個對大圖表都沒有重計算影響,必須用
這個才行。
\end{itemize}

\subsection{設定 counter}
排版中還有一些數字常常要加總,頁數,圖表數目,參照數目等等。
\begin{itemize}
\item \verb=\newcounter{mycounter}[section]= 命名一個 mycounter ,然後每次
      section counter 增加時,就會變成0
\item \verb=\setcounter{mycounter}{7}= 設 mycounter 7
\item \verb=\addtocounter{mycounter}{8}= mycounter 加8
\item \verb=\stepcounter{mycounter}= mycounter 加 1
\end{itemize}
counter 取值
\begin{center}
\begin{tabular}{ccc}
命令 & 效果 & 意義 \\
\hline
\verb=\thesection= & \thesection & section counter 值\\
\verb=\thepage= & \thepage &  page counter 值\\
\verb=\value{section}= & x & value 回傳不是字串,不能印 \\
\verb=\arabic{chapter}= & \arabic{chapter} &  印出 chapter counter 的阿拉伯符號\\
\verb=\roman{page}= & \roman{page} & 印出 page counter 的羅馬符號\\
\end{tabular}
\end{center}

\section{寫新macro}
\subsection{new}
使用 newcommand 來定義新函數。參數用 \#1 \#2 \#3 ...這樣
\begin{verbatim}
\newcommand {\name} [參數數目][default] { this is new command with #1 #2 ...}
\end{verbatim}
參數數目就是包含選擇性與必要性的參數數目總和,
default 就是內定值,就是之前用命令時,選擇性的命令,可給可不給。所以給一個例子:
\begin{flushleft}
\begin{tabular}{m{0.1\textwidth}|m{0.7\textwidth}}
命令 & 
\begin{verbatim}
\newcommand{\bighead}[2][大頭]{{#1}{#1}下雨不愁,{#2}有傘你有{#1}}
\bighead{\color{blue}人家}\newline\bighead{大雄}{小叮噹}
\end{verbatim}  \\
\hline \\[0.2cm]
效果 &
\newcommand{\bighead}[2][大頭]{{#1}{#1}下雨不愁,{#2}有傘你有{#1}}
\bighead{\color{blue}人家}\newline\bighead[大雄]{小叮噹} \\[0.2cm]
\hline \\[0.2cm]
&
由上面可以知道,它只是很像 C 的 macro ,字串完全代換而已,然後看得出來
使用\{\} 像是 shell, perl 可以解決很多事情。
\end{tabular}
\end{flushleft}
而新環境,在 default 後面有兩個中括號,before 裡面是會先執行的東西,
然後就是夾在 before end 中間的東西,最後執行 end 裡面的東西。
\begin{verbatim}
\newenvironment{name}[num][default]{before}{after}
\end{verbatim}
例子是
\begin{flushleft}
\begin{tabular}{m{0.1\textwidth}|m{0.7\textwidth}}
命令 & 
\begin{verbatim}
\newenvironment{mybox}
{\rule{1ex}{1ex}\hspace{\stretch{1}}}
{\hspace{\stretch{1}}\rule{1ex}{1ex}}
\begin{mybox}我的新盒子\end{mybox}
\end{verbatim}  \\
\hline \\[0.2cm]
效果 &
\newenvironment{mybox}
{\rule{1ex}{1ex}\hspace{\stretch{1}}}
{\hspace{\stretch{1}}\rule{1ex}{1ex}}
\begin{mybox}我的新盒子\end{mybox}
\end{tabular}
\end{flushleft}
一般程式都會有判斷,迴圈語法,\TeX{} 自然也提供,
hooks 是撞到一個條件時,去呼叫之前特別定義的函數。

% TODO plain tex and customized macro
%\subsection{plain \TeX{}}
%\subsection{custom macro}
\clearpage
