%      Copyright (c)  2023 Cyril Huang, Gyoza Workshop
%      Permission is granted to copy, distribute and/or modify this document
%      under the terms of the GNU Free Documentation License, Version 1.1
%      or any later version published by the Free Software Foundation;
%      with the Invariant Sections being LIST THEIR TITLES, with the
%      Front-Cover Texts being LIST, and with the Back-Cover Texts being LIST.
%      A copy of the license is included in the section entitled "GNU
%      Free Documentation License".

\documentclass[10pt,t]{beamer}
\usepackage{xeCJK}
\usepackage{xcolor}
\usepackage{listings}
\setmainfont{Noto Sans Mono}
\setCJKmainfont{Noto Sans CJK TC}

\usetheme{Warsaw}
\setbeamertemplate{background}{
  \includegraphics[width=\paperwidth,height=\paperheight]{images/bg.jpg}
}

\title{用 \LaTeX 做簡報}
\author{
  大頭
  \inst{1}
  \and
  大雄
  \inst{2}
}

\institute {
  \inst{1} 餃子出版社
  \and
  \inst{2} 餃子出版社
}

\begin{document}

\maketitle

\begin{frame}{大綱}
  \tableofcontents
\end{frame}

\section{簡介}
使用 beamer 文件,主要在 document 裡面多一個 frame 環境夾住一張簡報。如下
\begin{verbatim}
\documentclass[t]{beamer}
\usepackage{xeCJK}
\setCJKmainfont{Noto Sans CJK TC}

\title{簡介} \author {大頭}
\begin{document}
\maketitle
\begin{frame} \frametitle{大綱}
  \tableofcontents
\end{frame}

\section{第一段}
一
\subsection{次段}
次

\begin{frame} \frametitle{幻燈片標題一之一}
\end{frame}
\begin{frame} \frametitle{幻燈片標題一之二}
\end{frame}

\section{第二段}
\begin{frame} \frametitle{幻燈片標題二之一}
\end{frame}
\end{document}
\end{verbatim}

\begin{itemize}
\item 主要用 section, subsetion, frame 做層次的分門別類,
\item 在 section, subsection 內跟一般 \LaTeX 的用法一樣就可,
      也可以插圖表等等。它們的內容就會一直呈現出來像紙張大小變小的頁面。
\item frame 內定就只會出現一頁,如果長度太長,下面內容會被砍掉。
\item tableofcontents 也變成一張 frame 成為總綱,會自動編 section subsection。
\item frame 下也是很多用法一樣,只是內定是單張,所以像 verbatim 使用會不一樣,
      要注意。
\item 對齊選項 t, c, b 表示 top, center, bottom 對齊。內定是 center。
      可用在 documentclass 或單張 frame
\item bug: section 下用 verbatim, listings 會跑出一大段中間空白。
      解決方法就是盡量都放進一個 frame 中,不要用 section 寫東西,這也是對的。
\end{itemize}

\begin{frame}[fragile] \frametitle{frame 特殊選項}
  \begin{itemize}
  \item fragile 選項,如果有 verbatim, listings, 不管 inline 或環境的都要用
        fragile。
  \item plain 選項,所有 title headline .... 通通消失,就一張大白紙給你,
        通常用來插圖。
  \item bug: 就是在 frame 下面用 verbatim 時,如果裡面有 begin end frame 環境
        的語法,xelatex 會出錯。
  \end{itemize}
  \begin{center}
  \verb=\=begin\{frame\}\verb=[fragile]=\[\dots\]\verb=\=end\{frame\}
  \end{center}
\end{frame}

\section{動態效果 overlay}
\begin{frame}[fragile] \frametitle{用 pause}
光靜態 list 列舉是很簡單,動態效果就是把單張 frame 再切成更多小單元呈現。
在之前每一次右鍵就是下一張,左鍵就是上一張 slide。現在要把 frame
切割成更多單元,然後右鍵左鍵變成是上下一個單元而已。
\begin{verbatim}
\begin{itemize}
  \item 用 \verb=\pause= 來分割單元。
  \item 以上為第一單元,在每個呈現不管文字或圖表後加個 pause 就可以了。
  \pause
  \item 第二單元
\end{itemize}
不用 item 也可以 \\
\pause
哈哈
\end{verbatim}
\end{frame}

\begin{frame}[fragile]
\begin{itemize}
  \item 用 \verb=\pause= 來分割單元。
  \item 以上為第一單元,在每個呈現不管文字或圖表後加個 pause 就可以了。
  \pause
  \item 第二單元
\end{itemize}
不用 item 也可以 \\
\pause
哈哈
\end{frame}

\begin{frame}[fragile] \frametitle{用 bracket<> 來指定}
  \begin{itemize}
    \item \verb=\only<1>{文字\\}= 只出現在第一次。
    \item \verb=\uncover<2-5>{文字\\}= only 的反操作在第二到第五次右鍵都出現。
    \item \verb=\visible<1,4>{文字\\}= 出現在第一與第四次,
          跟 only 不一樣是會佔空間留下空白。
    \item \verb=\invisible<-3>{文字\\}= visible 的反操作。
          -3 表示從第3之前,3-表示第3之後所有。
  \end{itemize}
  例子
  \begin{verbatim}
  \only<1>{我是 only 第一次\\}
  \uncover<1-3>{我是 uncover 1-5\\}
  \visible<2,4> {我是visible第2次第4次\\}
  \invisible<5> {我是invisible第5次\\}
  \end{verbatim}
\end{frame}

\begin{frame}
  \only<1>{我是 only 第一次\\}
  \uncover<1-3>{我是 uncover 1-3\\}
  \visible<2,4> {我是visible第2次第4次\\}
  \invisible<5> {我是invisible第5次\\}
\end{frame}

\begin{frame}[fragile] \frametitle{用 itemize 或其他 text 命令}
直接用整個 itemize 環境
\begin{verbatim}
  \begin{itemize}[<+->]
    \item 在環境 itemize 直接用選項 <+-> 就會自動一次一個 item
    \item 不然可以用\verb=\item<2>= 跟上面一樣的 <> 規則。
    \item 在其他原本 {\LaTeX} 的 text 命令像 
          \verb=\color \textbf= 都可以。
  \end{itemize}
  \visible<4->{\color{blue}直接用 item <> 的例子}
  \begin{itemize}
    \item<4> {\color{blue}step 1}
    \item<5,7,9> {\color{red}step 2}
    \item<4-4> step 3
    \item<6> step 4
    \item<7,8> step 5
    \item<4-9> step 6
  \end{itemize}
\end{verbatim}
\end{frame}

\begin{frame}[fragile]
  \begin{itemize}[<+->]
    \item 在環境 itemize 直接用選項 <+-> 就會自動一次一個 item
    \item 不然可以用\verb=\item<2>= 跟上面一樣的 <> 規則。
    \item 在其他原本 {\LaTeX} 的 text 命令像 \verb=color textbf= 都可以。
  \end{itemize}
  \visible<4->{\color{blue}直接用 item <> 的例子}
  \begin{itemize}
    \item<4> {\color{blue}step 1}
    \item<5,7,9> {\color{red}step 2}
    \item<4-4> step 3
    \item<6> step 4
    \item<7,8> step 5
    \item<4-9> step 6
  \end{itemize}
\end{frame}

\section{columns}
\begin{frame}[fragile] \frametitle{使用 column}
除了用 table 外,其實是應該用 column 來切割幻燈片的,就像 powerpoint 的兩
個大物件分別在左右兩邊。很簡單沒有 row 。
\begin{verbatim}
  \begin{columns}[c]
    \column{.45\textwidth}
     第一欄位插個圖吧 \\[2cm]
     \includegraphics[width=0.8\textwidth]{images/sphinx.png}
    \column{.45\textwidth}
     第二個欄位就寫字
     \begin{itemize}
       \item 1
       \item 2
     \end{itemize}
  \end{columns}
\end{verbatim}
選項中一樣是對齊選項 t,c,b
\end{frame}

\begin{frame}
  \begin{columns}[c]
    \column{.45\textwidth}
     第一欄位插個圖吧 \\[2cm]
     \includegraphics[width=0.8\textwidth]{images/sphinx.png}
    \column{.45\textwidth}
     第二個欄位就寫字
     \begin{itemize}
       \item 1
       \item 2
     \end{itemize}
  \end{columns}
\end{frame}

\section{theme}
\begin{frame}[fragile] \frametitle{內定 theme}
內部已經有很多 theme 可以套用,都必須在 preamble 設定
  \begin{itemize}
  \item \verb=\usetheme{Warsaw}= 設定整個長相。
  \item \verb=\usecolortheme{beaver}= 設定配色
  \item \verb=\useoutertheme{infolines}= 設定標題跟腳註長相。
  \item \verb=\useinnertheme{rectangles}= 設定 item 前面的點點形狀。
\end{itemize}
\vfill
\begin{center}
\begin{tabular}{cp{0.5\textwidth}}
  選擇 theme & 參考選項\\
  \hline
  theme & \href{https://hartwork.org/beamer-theme-matrix/}
	{{\color{blue}theme matrix}}\\
  outertheme & 8 種 infolines shadow smoothbars	split miniframes sidebar smoothtree tree\\
  innertheme & 四種 rectangles inmargin circles rounded\\
\end{tabular}
\end{center}
\end{frame}

\begin{frame}[fragile] \frametitle{自訂 theme}
最簡單就是貼一張公司的 background 圖像或者網路上找背景圖設在最前面
\begin{verbatim}
\setbeamertemplate{background}{
  \includegraphics[width=\paperwidth,height=\paperheight]{bg.jpg}
}
\end{verbatim}
其他的比較複雜要一個個選項去設定, 要去看
\href{https://tug.ctan.org/macros/latex/contrib/beamer/doc/beameruserguide.pdf}
 {\color{blue}beamer 的使用手冊},
有詳細的細部說明,主要用 setbeamercolor setbeamertemplate去設定。
  \begin{itemize}
  \item \verb=\setbeamercolor=
  \item \verb=\setbeamertemplate=
  \end{itemize}
\end{frame}

\end{document}
